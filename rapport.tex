\documentclass{article}
\usepackage[utf8]{inputenc}
\usepackage[french]{babel}
\usepackage{mathtools}
\usepackage{graphicx} 
\usepackage{hyperref}
\usepackage{float}
\usepackage{listings}
\usepackage{xcolor}

\title{Rapport projet PFA 2018-2019}
\date{\today}
\author{Lam NGUYEN THIET
\and Kenyu KOBAYASHI}
\begin{document}
\maketitle
\section{Introduction}
Ce projet implémente un jeu dans le langage OCAML, en utilisant les fonctionalités fonctionelles ( en majorité ), impératives, orientée objet.

\section{Le jeu}
\subsection{But}
Le but du jeu est de détruire les factions ennemies. Pour se faire, il suffit de tuer toutes leur unités.

\subsection{Factions}
Il y a 3 factions dans le jeu. Les cagoulés, les vert et les bleus. Vous contrôlez les bleus.

\subsection{Unités}
Il existe deux types d'unités.

\begin{itemize}
    \item Le soldat est l'unité de base. Elle peut se déplacer, attaquer et récupérer des items.
    \item La ville permet de faire apparaître des soldats. Cette unité est très importante car si on la perd, on ne peut plus produire de soldat.
\end{itemize}

\subsection{Items}
Il existe deux items.

\begin{itemize}
    \item Le pack de soin régénére la vie des unités. Il est possible d'avoir plus de points de vie que l'on avait au départ.
    \item La bombe nucléaire détruit tout dans un rayon de 3 cases.
\end{itemize}

\subsection{Plateau du jeu}
Le plateau est une grille d'hexagone. La carte est une île deserte dans l'océan avec des biômes variés.

\subsubsection{Type de cases}
Il y a 3 types de cases. A l'heure actuelle, elle ne sont que cosmétiques, par la suite elle peuvent avoir un impact sur l'efficacité de combat de tel unité de
tel faction, mais par manque de temps je n'ai pas pu les faire.

Il y a la neige, le desert et l'herbe.

\subsubsection{Caractèristiques du terrain}
En plus du biome, il y a des caractèristiques sur le terrain pour chaque case.

\begin{itemize}
    \item Les forêts et les collines coûtent plus cher pour le movement.
    \item Les montagnes et les lacs sont des obstacles ne peuvent pas être traversés.
\end{itemize}

\subsection{Tour par tour}
Le jeu se déroule en tour par tour, similaire au jeu \textit{Civilization}. C'est d'ailleurs
sur quoi je me suis inspiré pour le jeu.

\subsubsection{Mouvement}

Chaque unité a un nombre de movement. Lorsqu'il se déplace d'une case, il consomme $n$ points selon la case sur laquelle il atterit.
Dans le jeu, les soldats sont les seuls à pouvoir se déplacer. Les villes ne peuvent pas.

\subsubsection{Attaque}
Pour tuer les autres unités il faut les attaquer. Encore une fois de manière analogue à \textit{Civilization},
les unités disposent d'une force d'attaque et d'une force de défense.

Par la suite, $src$ et $dst$ représente respectivement l'unité qui attaque et l'unité
qui défend.

Lorsque deux unités s'attaquent, les nouveaux point de vie se calculent de cette manière : 

\begin{align}
    healthpoints_{dst,new} &= max \{0,healthpoints_{dst,old} - strength_{attack,src}\} \\
    healthpoints_{src,new} &= max \{0,healthpoints_{src,old} - strength_{defense,dst}\}
\end{align}

Et leurs nouvelles positions : 
\begin{align}
     x_{dst,new},y_{dst,new} &=
     \begin{cases}
        \texttt{null},\texttt{null} & \text{si } healthpoints_{dst_,new} = 0 \text{,}\\
        x_{dst,old},y_{dst,old} & \text{sinon.} 
     \end{cases} \\
     x_{src,new},y_{src,new} &=
     \begin{cases}
        \texttt{null},\texttt{null} & \text{si } healthpoints_{src,new} = 0 \text{,}\\
        x_{dst,old},y_{dst,old} & \text{si } healthpoints_{dst_,new} = 0 \text{,}\\
        x_{src,old},y_{src,old} & \text{sinon.} 
     \end{cases}
\end{align}

\subsection{Intelligence Artificielle}
Les IA sont assez simple. De base, elles se déplacent au hasard.
Si elles rencontrent un ennemi, elle va se diriger vers cet ennemi en priorité. Si il y a un
autre ennemi elles s'attaquent, même si ce n'était pas l'ennemi en priorité.

En dessous d'un certain seuil, les IA vont chercher à fuire et chercher un pack de soin. Mais si il y a un ennemi
tout prés, elles vont se suicider et attaquer cet ennemi, car elles savent qu'elles vont mourir et vont préférer attaquer 
pour donner une chance à leur alliés.

Sinon, en mode patrouille, si elles voient une bombe nucléaire, elles la prennent et l'utilise sur un ennemi au hasard.
\section{Répartition du travail}

\section{Développement du jeu}
Chaque phase sera développée plus en profondeur dans la suite du rapport.
Chaque implémentation est listé par ordre chronologique dans le développement.

\subsection*{Phase 1 : Fondation}
\begin{enumerate}
    \item Familiarisation avec \texttt{SDL} et factorisation de code qui était assez récurrent.
    \item Implémentation de la boucle principale
    \item Généralisation du type \texttt{context} et sa mise à jour
\end{enumerate}

\subsection*{Phase 2 : Instances de jeu}
\begin{enumerate}
    \item Définition des instances du jeu (la boucle du menu, du jeu etc. )
    \item Création de boutons temporaires pour lancer le jeu et le quitter si on appuie sur la croix
\end{enumerate}


\subsection*{Phase 3 : Plateau de jeu}
\begin{enumerate}
    \item Définition du plateau de jeu (représenté par une matrice)
    \item Implémentation des fonctionalités de la grille d'hexagone
    \item Définition des tuiles et des \texttt{enum} qui la définissent
    \item Les dessins du plateau (tuiles, forêts, etc.)
\end{enumerate}

\subsection*{Phase 4 : Unités}
\begin{enumerate}
    \item Définition des unités
    \item Définition des constantes qui les définissent
    \item Les dessins des unités
\end{enumerate}

\subsection*{Phase 5 : Actions}
\begin{enumerate}
    \item Formalisation et généralisation des actions et leur retour pour qu'ils puissent tous être du même type
    \item Implémentation des actions de déplacement et attaques
    \item Interaction temporaire avec les unités avec le clavier
    \item Interaction du retour des actions avec le système de contexte
\end{enumerate}

\subsection*{Phase 6 : IA}
\begin{enumerate}
    \item Formalisation d'un comportement d'une unité 
    \item Systèmes similaire aux automates d'états finis pour sélectionner le comportement de chaque unités selon son environment
    \item IA qui se déplacement au hasard,
    \item et attaque une cible si il y en a une qui se trouve à proximité,
    \item et qui va chercher des packs de soin si elles n'a pas beaucoup de points de vie,
    \item et qui va chercher les bombes nucléaires si elle peut.
\end{enumerate}

\subsection*{Phase 7 : Interfaces}
\begin{enumerate}
    \item Affichage des informations liés à chaque unités (ses points de vie et points de mouvement)
    \item Système d'interface avec des \textit{event listeners}
    \item Formalisation et généralisation des interactions avec les interfaces dans le contexte
\end{enumerate}

\end{document}